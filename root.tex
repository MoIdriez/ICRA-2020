%%%%%%%%%%%%%%%%%%%%%%%%%%%%%%%%%%%%%%%%%%%%%%%%%%%%%%%%%%%%%%%%%%%%%%%%%%%%%%%%
%2345678901234567890123456789012345678901234567890123456789012345678901234567890
%        1         2         3         4         5         6         7         8

\documentclass[letterpaper, 10 pt, conference]{ieeeconf}  % Comment this line out if you need a4paper

%\documentclass[a4paper, 10pt, conference]{ieeeconf}      % Use this line for a4 paper

\IEEEoverridecommandlockouts                              % This command is only needed if 
                                                          % you want to use the \thanks command

\overrideIEEEmargins                                      % Needed to meet printer requirements.
\usepackage{slashbox}
%In case you encounter the following error:
%Error 1010 The PDF file may be corrupt (unable to open PDF file) OR
%Error 1000 An error occurred while parsing a contents stream. Unable to analyze the PDF file.
%This is a known problem with pdfLaTeX conversion filter. The file cannot be opened with acrobat reader
%Please use one of the alternatives below to circumvent this error by uncommenting one or the other
%\pdfobjcompresslevel=0
%\pdfminorversion=4

% See the \addtolength command later in the file to balance the column lengths
% on the last page of the document

% The following packages can be found on http:\\www.ctan.org
%\usepackage{graphics} % for pdf, bitmapped graphics files
%\usepackage{epsfig} % for postscript graphics files
%\usepackage{mathptmx} % assumes new font selection scheme installed
%\usepackage{times} % assumes new font selection scheme installed
%\usepackage{amsmath} % assumes amsmath package installed
%\usepackage{amssymb}  % assumes amsmath package installed

\title{\LARGE \bf
A framework for comparing mobile robot navigation
}


\author{Mohamed Osama Idries, Matthias Rolf, and Tjeerd V. olde Scheper% <-this % stops a space
\thanks{M.O. Idries, M. Rolf, and T. Scheper are with Faculty of Technology, Design and Environment,
        Oxford Brookes University, Headington Rd, Oxford OX3 0BP, The United Kingdom
        {\tt\small {midries,mrolf,tvolde-scheper}@brookes.ac.uk}}%
}


\begin{document}



\maketitle
\thispagestyle{empty}
\pagestyle{empty}


%%%%%%%%%%%%%%%%%%%%%%%%%%%%%%%%%%%%%%%%%%%%%%%%%%%%%%%%%%%%%%%%%%%%%%%%%%%%%%%%
\begin{abstract}
	Exploration and navigation are one of the fundamental problems in mobile robotics. Efforts to address these range from reactive, map-based to learning-based approaches. With each method being developed and tested in an isolated environments, the precise improvements of these methods are unknown. This paper presents a framework to simulate, evaluate and compare these different algorithms.
	********************
	Our results demonstrate how methods compare over a range of attributes and environments. We anticipate that this framework and findings allow for development of more advanced approaches, but also serve as a good step towards navigating dynamic environments. 
	
\end{abstract}


%%%%%%%%%%%%%%%%%%%%%%%%%%%%%%%%%%%%%%%%%%%%%%%%%%%%%%%%%%%%%%%%%%%%%%%%%%%%%%%%
\section{Introduction}

Introduction to what exploration and navigation is.
 
Where is it used. Why is it of importance.

Introduce the key issue in the dynamic environments.

Introduce main contribution of this paper to aid in solving the problem above

Lorem ipsum dolor sit amet, consectetur adipiscing elit. Mauris sed venenatis nunc. Mauris at orci sodales, placerat massa in, gravida ligula. Duis nisi magna, fermentum sit amet sem ut, lacinia tincidunt mauris. Praesent et bibendum est. Ut faucibus orci lorem, rutrum laoreet orci posuere non. Donec in urna sed ante iaculis tempus vitae id justo. Maecenas id tincidunt neque. Ut vitae imperdiet augue.

Lorem ipsum dolor sit amet, consectetur adipiscing elit. Mauris sed venenatis nunc. Mauris at orci sodales, placerat massa in, gravida ligula. Duis nisi magna, fermentum sit amet sem ut, lacinia tincidunt mauris. Praesent et bibendum est. Ut faucibus orci lorem, rutrum laoreet orci posuere non. Donec in urna sed ante iaculis tempus vitae id justo. Maecenas id tincidunt neque. Ut vitae imperdiet augue.

Lorem ipsum dolor sit amet, consectetur adipiscing elit. Mauris sed venenatis nunc. Mauris at orci sodales, placerat massa in, gravida ligula. Duis nisi magna, fermentum sit amet sem ut, lacinia tincidunt mauris. Praesent et bibendum est. Ut faucibus orci lorem, rutrum laoreet orci posuere non. Donec in urna sed ante iaculis tempus vitae id justo. Maecenas id tincidunt neque. Ut vitae imperdiet augue.

Lorem ipsum dolor sit amet, consectetur adipiscing elit. Mauris sed venenatis nunc. Mauris at orci sodales, placerat massa in, gravida ligula. Duis nisi magna, fermentum sit amet sem ut, lacinia tincidunt mauris. Praesent et bibendum est. Ut faucibus orci lorem, rutrum laoreet orci posuere non. Donec in urna sed ante iaculis tempus vitae id justo. Maecenas id tincidunt neque. Ut vitae imperdiet augue.

\begin{figure}[thpb]
	\centering
	\framebox{\parbox{3in}{This will contain a graphic showing all the different algorithm in a single environment. Each approach has it's own color. This serves to demonstrate in a simple way how the different algorithms behave in a simple and attracting way. Lorem ipsum dolor sit amet, consectetur adipiscing elit.Lorem ipsum dolor sit amet, consectetur adipiscing elit.Lorem ipsum dolor sit amet, consectetur adipiscing elit.
	}}
	%\includegraphics[scale=1.0]{figurefile}
	\caption{Different navigational algorithms path towards a goal}
	\label{figurelabel}
\end{figure}


Lorem ipsum dolor sit amet, consectetur adipiscing elit. Mauris sed venenatis nunc. Mauris at orci sodales, placerat massa in, gravida ligula. Duis nisi magna, fermentum sit amet sem ut, lacinia tincidunt mauris. Praesent et bibendum est. Ut faucibus orci lorem, rutrum laoreet orci posuere non. Donec in urna sed ante iaculis tempus vitae id justo. Maecenas id tincidunt neque. Ut vitae imperdiet augue.

Lorem ipsum dolor sit amet, consectetur adipiscing elit. Mauris sed venenatis nunc. Mauris at orci sodales, placerat massa in, gravida ligula. Duis nisi magna, fermentum sit amet sem ut, lacinia tincidunt mauris. Praesent et bibendum est. Ut faucibus orci lorem, rutrum laoreet orci posuere non. Donec in urna sed ante iaculis tempus vitae id justo. Maecenas id tincidunt neque. Ut vitae imperdiet augue.

Lorem ipsum dolor sit amet, consectetur adipiscing elit. Mauris sed venenatis nunc. Mauris at orci sodales, placerat massa in, gravida ligula. Duis nisi magna, fermentum sit amet sem ut, lacinia tincidunt mauris. Praesent et bibendum est. Ut faucibus orci lorem, rutrum laoreet orci posuere non. Donec in urna sed ante iaculis tempus vitae id justo. Maecenas id tincidunt neque. Ut vitae imperdiet augue.



\section{Related work and background}

Present the main navigational algorithmic advancements justifying your choices too
Present the comparison papers and how they have previously approached these comparisons


\section{Methodology}
\subsection{Experimental setup}
\subsection{Evaluation methods}

\section{Navigational algorithms}

\subsection{random walk}
\subsection{wall follower}
\subsection{pheromone potential field}
\subsection{A* algorithm}
\subsection{Q learning}


\section{Results}

For each trial (Robot, Goal, Map) we have the following measurements;
\begin{itemize}
	\item \textit{$d_{g}$} distance to goal
	\item \textit{$\epsilon$} success 
	\item \textit{$d_{s}$} distance to starting point
	\item $\sigma$ area explored
	\item \textit{c} path cost
	\item $\mu$ computational overhead
	\item \textit{$m_{s}$} map size
	\item \textit{$m_{d}$} map density
\end{itemize}

- How do we determine map complexity (needs a factor cross maps so that bigger maps are considered more complex)
$$
map_{complexity} = \frac{d_{eu}}{d_{worst}} *  \frac{map_{den}}{map_{size}} \eqno{(*)}
$$

$$
map_{complexity} = \frac{d_{eu}}{d_{worst}} *  \frac{map_{den}}{map_{size}} \eqno{(*)}
$$

- How does an algorithm perform on average 
$$
algorithm_{score} = (1 - \frac{d_{g}}{d_{eu}}) * \frac{d_{b} * c}{d_{eu}^2} \eqno{(*)}
$$
\begin{table}[ht!]
	\centering
	\begin{tabular}{ *{5}{|c}|} 
		\hline
		\backslashbox[10mm]{algorithm}{complexity} & 0-25\% & 25-50\% & 50-75\% & 75-100\% \\
		\hline
		random walk& 0 & 0 & 0 & 0 \\
		wall follower& 0 & 0 & 0 & 0 \\
		pheromone potential field& 0 & 0 & 0 & 0 \\
		A* algorithm& 0 & 0 & 0 & 0 \\
		Q learning& 0 & 0 & 0 & 0 \\
		\hline
	\end{tabular}
\end{table}


- How does an algorithm's performance translate to hybrid maps.

\begin{table}[ht!]
	\centering
	\begin{tabular}{ *{8}{|c}|} 
		\hline
		\backslashbox[10mm]{algorithm}{type} & O & D & H & OAD & OAH & DAH & ODH \\
		\hline
		random walk& 0 & 0 & 0 & 0 & 0 & 0 & 0 \\
		wall follower& 0 & 0 & 0 & 0 & 0 & 0 & 0  \\
		pheromone potential field& 0 & 0 & 0 & 0 & 0 & 0 & 0 \\
		A* algorithm& 0 & 0 & 0 & 0 & 0 & 0 & 0 \\
		Q learning& 0 & 0 & 0 & 0 & 0 & 0 & 0 \\
		\hline
	\end{tabular}
\end{table}
\section{Discussion and conclusion}


\addtolength{\textheight}{-12cm}   % This command serves to balance the column lengths
                                  % on the last page of the document manually. It shortens
                                  % the textheight of the last page by a suitable amount.
                                  % This command does not take effect until the next page
                                  % so it should come on the page before the last. Make
                                  % sure that you do not shorten the textheight too much.

%%%%%%%%%%%%%%%%%%%%%%%%%%%%%%%%%%%%%%%%%%%%%%%%%%%%%%%%%%%%%%%%%%%%%%%%%%%%%%%%



%%%%%%%%%%%%%%%%%%%%%%%%%%%%%%%%%%%%%%%%%%%%%%%%%%%%%%%%%%%%%%%%%%%%%%%%%%%%%%%%



%%%%%%%%%%%%%%%%%%%%%%%%%%%%%%%%%%%%%%%%%%%%%%%%%%%%%%%%%%%%%%%%%%%%%%%%%%%%%%%%






\end{document}
